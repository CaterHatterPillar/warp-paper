% GAME_HARD_01_ABSTRACT.tex

% ABSTRACT
\begin{abstract}
A convenient approach toward more transparent debugging and profiling of GPU-accelerated applications is to simulate GPU-bound workloads on CPUs.
This approach is also applicable in situations where target hardware is simply not available, as is often the case with server-side applications, or would require too many system resources to initialize.
However, when simulating GPU-bound workloads on CPUs, one may experience severe performance losses due to computational overhead.
Consequently, the subject of this study is performance variations between three device drivers of the DirectX framework; using DirectCompute and the high speed software rasterizer WARP (Windows~Advanced~Rasterization~Platform).
The performance of WARP is compared to that of traditional GPU hardware acceleration and the standard driver for software rasterization - the Reference~Device~Driver.
Experimental results show a major performance boost when compared to that of software rasterization using the reference~device~driver, indicating that performance losses traditionally obstructing simulation of throughput-oriented workloads on CPUs may be sufficiently amended by technologies, such as WARP, to the degree that simulation may be considered viable for extended use above and beyond that of current in-development utilities.
\end{abstract}

