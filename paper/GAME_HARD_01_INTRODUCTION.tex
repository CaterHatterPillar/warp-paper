% GAME_HARD_01_INTRODUCTION.tex

% INTRODUCTION
\section{INTRODUCTION}
\label{sec:introduction}
When developing GPU-bound applications, careless additions to  program execution, such as branching, register use, or memory accesses, may induce substantial performance obstacles due to massively parallelized instruction sets and architectural differences between on-chip hardware (see Performance Considerations by Kirk~and~Hwu~\cite[ch.~6]{Kirk:2010:PMP:1841511} for an analysis on the volatility of GPU performance).
The increased utilization of complex GPU-kernels has brought forth the need of more extensive debugging- and profiling-options involving access of data that may be difficult to retrieve from hardware.
Therefore, it may be desirable for developers to access data being computed on the graphics card - a possibility often limited in terms of GPGPU (General-Purpose computing on Graphics Processing Units) technologies, possibly due to architectural differences between chip manufacturers.
A preferred solution to this problem is to simulate GPU workloads on CPUs~\citeweb[]{drivertypes}, such as described by Kerr~et~al.~\cite[p.~416-419]{Hwu:2011:GCG:2103614} concerning the implementation of the GPU~Ocelot compilation framework~\citeweb[]{gpuocelot}, often in exchange for substantial performance implications.
Other reasons to simulate GPU-kernels may concern pre-silicon development - that is, development for hardware not yet existent, when hardware is busy, or otherwise unavailable~\citeweb[]{warp}.

This paper comprises an investigation into the performance of simulated GPGPU-kernels, in relation to hardware acceleration, by the means of analyzing several software rasterizers.
Accordingly, the material concerns inquiry into the DirectCompute-framework on the Windows-platform, analyzing the performance of a GPGPU-kernel on-chip, using the DirectX standard software rasterizer, and utilizing the DirectX~11 addition Microsoft~WARP which promises high-speed software rasterization~\citeweb[]{warp}.
The study concludes that performance losses inflicted by software rasterization may be sufficiently amended for simulation of GPU-bound workloads to be considered viable for extended use, as originally proposed by Microsoft~\citeweb[]{warp}.
Thus, this paper proposes use of WARP-like technologies if graphics hardware is unavailable, not sufficient, or busy, and to act as an extension of the reference~device~driver as means to verify hardware- or driver-bound errors for complex GPU programs.
As such, the study concerns the fields of simulation and GPU-technologies, with the purpose of facilitating debugging and profiling of GPU-kernels, whilst maintaining acceptable performance.
The remainder of this document presents the method and process to acquire the data used, the technologies using which it has been acquired, the results in and of their own, the conclusions based off the results and finally; personal reflections from the author with propositions of further study.
