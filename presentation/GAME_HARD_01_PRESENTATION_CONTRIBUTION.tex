% GAME_HARD_01_PRESENTATION_CONTRIBUTION.tex
\section{Contribution}

% EXPERIMENT
\subsection{Experiment}
\begin{frame}
\frametitle{EXPERIMENT}

$200\times 200$ matrix multiplication using Direct3D DirectCompute:

\begin{enumerate*}
\item Randomize matrices $A$ and $B$
\item Establish $AB=Ref$
\item Start high-precision timer
\item Calculate $AB=C$ using Direct3D
\item Stop high-precision timer
\item Verify $C$ and $Ref$ deviation
\end{enumerate*}

\end{frame}

% DIRECTCOMPUTE KERNELS
\subsection{DirectCompute Kernels}
\begin{frame}[t]
\frametitle{DIRECTCOMPUTE KERNELS}

\begin{columns}[T]
  \begin{column}[T]{0.5\textwidth}
    \begin{center}1.\end{center}
    Matrix mult. w. Thread Blocks:
    \begin{itemize}
    \item Invoked once per entry
    \item Performed in blocks of $16\times 16$
    \end{itemize}
  \end{column}
  \begin{column}[T]{0.5\textwidth}
    \begin{center}2.\end{center}
    Matrix mult. w. Thread Blocks and Shared Memory:
    \begin{itemize}
    \item Utilizes shared memory to cache matrices
    \end{itemize}
  \end{column}
\end{columns}

{\footnotesize\it%
  \begin{block}{}
    \begin{columns}[T]
      \begin{column}[T]{0.33\textwidth}
        Constant Memory:
        \begin{itemize}
        \item Latency- and bandwidth oriented memory
        \item Read-only device access
        \end{itemize}
      \end{column}
      \begin{column}[T]{0.33\textwidth}
        Global Memory:
        \begin{itemize}
        \item Device main memory
        \item Read- and write access by device host
        \end{itemize}
      \end{column}
      \begin{column}[T]{0.33\textwidth}
        Shared Memory:
        \begin{itemize}
        \item High-speed device read-/write memory
        \item May be accessed across 'blocks'
        \end{itemize}
      \end{column}
    \end{columns}
  \end{block}
}

\end{frame}

% TEST CASES
\subsection{Test Cases}
\begin{frame}
\frametitle{TEST CASES}

\begin{block}{1. Hardware-Acceleration}
  \begin{itemize}
  \item Execution on a modern video card
  \item Maximising FLOPS output
  \item First reference measurement for the performance of WARP
  \end{itemize}
\end{block}
\noindent\rule{11cm}{0.4pt}
\begin{block}{2. Software Rasterization}
  \begin{itemize}
  \item Traditional software rasterization using the reference device driver
  \item Second reference measurement for the performance of WARP
  \end{itemize}
\end{block}
\begin{block}{3. Windows Advanced Rasterization Platform}
  \begin{itemize}
  \item Software rendering using WARP
  \end{itemize}
\end{block}

\end{frame}

% SOFTWARE
\subsection{Software}
\begin{frame}
\frametitle{SOFTWARE}

\begin{columns}[T]
  \begin{column}{0.33\textwidth}
    \textbf{matrixgen}\hfill$\rightarrow$\hfill
    \begin{itemize}
    \item Generates $A$ \& $B$
    \item Establishes $Ref$
    \item C$++$ and $C±±$~AMP
    \end{itemize}
  \end{column}
  \begin{column}{0.33\textwidth}
    \textbf{experiment}\hfill$\rightarrow$\hfill
    \begin{itemize}
    \item Produces $C$ using Direct3D
    \item Performs $100$ tests for each scenario:
      \begin{itemize}
      \item Kernel
      \item Precision\footnotemark
      \item Acceleration
      \end{itemize}
    \item C$++$ and HLSL
    \end{itemize}
  \end{column}
  \begin{column}{0.33\textwidth}
    \textbf{analytics}
    \begin{itemize}
    \item Analyzes raw data
    \item Compiles measurement average:
      \begin{itemize}
      \item Min/Max
      \item Avg
      \item StD
      \end{itemize}
    \item C$++$
    \end{itemize}
  \end{column}
\end{columns}

\footnotetext[1]{We perform the experiment in both integer and floating point precision.}

\end{frame}
